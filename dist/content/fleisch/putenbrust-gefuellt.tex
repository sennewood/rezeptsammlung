\section*{Gefüllte Putenbrust}
\addcontentsline{toc}{section}{Gefüllte Putenbrust}

\bigbreak
\rule{\textwidth}{0.4pt}

\subsection*{Zutaten}

\setlength{\columnseprule}{0pt}
\setlength{\columnsep}{1.5em}
\begin{multicols}{2}

    \begin{description}[align=right,leftmargin=!,labelwidth=\widthof{\bfseries xxPrisen}]
        \item[1kg] Putenbrust
        \item[125g] Mozarella
        \item[50g] Getrocknete Tomaten
        \item[4 EL] Olivenöl
        \item[400g] Kirschtomaten
        \item[2] Knoblauchzehen
    \end{description}

\columnbreak

    \begin{description}[align=right,leftmargin=!,labelwidth=\widthof{\bfseries xxPrisen}]
        \item[125ml] Wasser
        \item[$\frac{1}{2}$ TL] Gemüsebrühe
        \item[*] Zahnstocher
        \item[*] Küchengarn
        \item[Gewürze] Basilikum, Oregano, Salz, Pfeffer
        \item[Beilage] Nudeln oder Reis
    \end{description}

\vspace*{\fill}
\end{multicols}


\rule{\textwidth}{0.4pt}


\subsection*{Vorbereitungen}

\begin{itemize}
    \item Backofen vorheizen: Umluft, 150°C.
    \item Kräuter waschen.
    \item Tomaten in Stücke schneiden.
    \item Mozarella in Scheiben schneiden.
\end{itemize}



\bigbreak
\subsection*{Zubereitung}

\begin{itemize}
    \item Jeweils eine tiefe Tasche in die Putenbrust schneiden.
    \item Mozarella, Kräuter und getrocknete Tomaten in die Taschen füllen.
    \item Taschen mit Zahnstochern verschließen und mit Küchengarn umwickeln.
    \item Taschen in eine Fettpfanne legen und mit Öl bestreichen.
    \item Mit Salz und Pfeffer würzen.
    \item Im vorgeheizten Backofen \textbf{75 Minuten} garen.
    \item Knoblauch pressen.
    \item Tomaten und Knoblauch nach \textbf{50 Minuten} garen auf die Fettpfanne verteilen.
    \item Heißes Wasser mit der Brühe verrühren.
    \item Gemüse und Braten nach \textbf{60 Minuten} garen ablöschen.
    \item Ofen ausschalten und die restliche Zeit abwarten.
\end{itemize}

% \bigbreak

% \begin{itemize}
%     \item XXX
% \end{itemize}
